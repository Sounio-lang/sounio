% Quantitative Type Theory for Sounio
% Formal Specification - Day 51: L0 Foundations
%
% This document specifies the mathematical foundations for:
% 1. Quantitative Type Theory (QTT) with multiplicities
% 2. Erasure semantics for ontological types
% 3. Runtime representation modes for epistemic types

\documentclass[11pt]{article}
\usepackage{amsmath,amssymb,amsthm}
\usepackage{stmaryrd}
\usepackage{proof}
\usepackage{xcolor}

\newtheorem{definition}{Definition}
\newtheorem{theorem}{Theorem}
\newtheorem{lemma}{Lemma}
\newtheorem{corollary}{Corollary}

\newcommand{\mult}{\rho}
\newcommand{\zero}{\mathbf{0}}
\newcommand{\one}{\mathbf{1}}
\newcommand{\many}{\omega}
\newcommand{\Knowledge}{\mathsf{Knowledge}}
\newcommand{\Full}{\mathsf{Full}}
\newcommand{\Compact}{\mathsf{Compact}}
\newcommand{\Erased}{\mathsf{Erased}}

\title{Quantitative Type Theory for Sounio\\
\large Formal Specification v0.51}
\author{Sounio Language Team}
\date{Day 51: L0 Foundations}

\begin{document}
\maketitle

\begin{abstract}
This specification defines the formal foundations for Sounio's type system
extensions that enable 15 million ontological types with zero runtime overhead.
We extend classical type theory with \emph{multiplicities} from Quantitative
Type Theory (QTT) and define precise erasure semantics that guarantee
ontological types exist only at compile-time.
\end{abstract}

\section{Introduction}

Sounio integrates 15M+ ontological types (from SNOMED-CT, ICD-10, etc.)
into its type system. The key challenge is ensuring these types provide
compile-time safety without runtime overhead. We achieve this through:

\begin{enumerate}
\item \textbf{Multiplicities}: Each type binding is annotated with how many
      times it can be used (0, 1, or $\omega$).
\item \textbf{Erasure}: Types with multiplicity 0 are completely erased
      after type checking.
\item \textbf{Tiered Representation}: Runtime epistemic types have three
      representation modes based on usage patterns.
\end{enumerate}

\section{Multiplicities}

\subsection{The Multiplicity Semiring}

\begin{definition}[Multiplicity]
A multiplicity $\mult \in \{\zero, \one, \many\}$ represents how many times
a value can be used:
\begin{itemize}
\item $\zero$: Erased at runtime (compile-time only)
\item $\one$: Used exactly once (linear)
\item $\many$: Used any number of times (unrestricted)
\end{itemize}
\end{definition}

\begin{definition}[Semiring Operations]
Multiplicities form a semiring $(\{\zero, \one, \many\}, +, \times, \zero, \one)$:

\textbf{Addition} (sequential use):
\begin{align*}
\zero + \mult &= \mult \\
\mult + \zero &= \mult \\
\one + \one &= \many \\
\many + \mult &= \many
\end{align*}

\textbf{Multiplication} (nested use):
\begin{align*}
\zero \times \mult &= \zero \\
\mult \times \zero &= \zero \\
\one \times \mult &= \mult \\
\mult \times \one &= \mult \\
\many \times \many &= \many
\end{align*}
\end{definition}

\begin{theorem}[Semiring Laws]
The multiplicity operations satisfy:
\begin{enumerate}
\item Associativity: $(\mult_1 + \mult_2) + \mult_3 = \mult_1 + (\mult_2 + \mult_3)$
\item Commutativity: $\mult_1 + \mult_2 = \mult_2 + \mult_1$
\item Identity: $\zero + \mult = \mult$ and $\one \times \mult = \mult$
\item Annihilation: $\zero \times \mult = \zero$
\item Distribution: $\mult_1 \times (\mult_2 + \mult_3) =
      (\mult_1 \times \mult_2) + (\mult_1 \times \mult_3)$
\end{enumerate}
\end{theorem}

\subsection{Quantified Types}

\begin{definition}[Quantified Type]
A quantified type is a pair $\mult \cdot \tau$ where $\mult$ is a multiplicity
and $\tau$ is a type. We write:
\[
\zero \cdot \tau \quad \text{(erased)} \qquad
\one \cdot \tau \quad \text{(linear)} \qquad
\many \cdot \tau \quad \text{(unrestricted)}
\]
\end{definition}

\subsection{Typing Rules}

The key typing judgment is:
\[
\Gamma \vdash_\mult e : \tau
\]
meaning ``under context $\Gamma$, expression $e$ has type $\tau$ and uses
its resources with multiplicity $\mult$''.

\textbf{Variable Rule}:
\[
\infer[\text{Var}]{x : \mult \cdot \tau \vdash_\mult x : \tau}{}
\]

\textbf{Abstraction Rule}:
\[
\infer[\text{Abs}]{\Gamma \vdash_\mult \lambda x. e : (\mult' \cdot \sigma) \to \tau}
      {\Gamma, x : \mult' \cdot \sigma \vdash_\mult e : \tau}
\]

\textbf{Application Rule}:
\[
\infer[\text{App}]{\Gamma_1 + \mult \cdot \Gamma_2 \vdash_{\mult_1} e_1\,e_2 : \tau}
      {\Gamma_1 \vdash_{\mult_1} e_1 : (\mult \cdot \sigma) \to \tau &
       \Gamma_2 \vdash_{\mult_2} e_2 : \sigma}
\]

\section{Ontological Type Erasure}

\subsection{Erasure Categories}

\begin{definition}[Erasure Category]
We distinguish the following erasure categories:
\begin{itemize}
\item $\mathcal{E}_{\text{type}}$: Type-level computations
\item $\mathcal{E}_{\text{onto}}$: Ontological type annotations
\item $\mathcal{E}_{\text{proof}}$: Proof terms and witnesses
\item $\mathcal{E}_{\text{phantom}}$: Phantom type parameters
\end{itemize}
\end{definition}

\subsection{Ontological Types as Zero-Multiplicity}

\begin{definition}[Ontological Type]
An ontological type $O[\text{iri}]$ where $\text{iri}$ is an IRI from an
ontology (SNOMED, ICD-10, etc.) always has multiplicity $\zero$:
\[
O[\text{iri}] : \zero \cdot \text{Type}
\]
\end{definition}

\begin{theorem}[Ontological Erasure]
For any program $P$ containing ontological type annotations $O_1, \ldots, O_n$:
\[
\llbracket P \rrbracket_{\text{runtime}} = \llbracket P[\bot/O_1, \ldots, \bot/O_n] \rrbracket_{\text{runtime}}
\]
That is, ontological types can be replaced with a placeholder without
affecting runtime behavior.
\end{theorem}

\subsection{Subsumption Erasure}

Subsumption checks of the form $\tau_1 <: O[\text{iri}]$ are performed at
compile time and erased:

\[
\infer[\text{Onto-Sub}]{\Gamma \vdash e : O[\text{iri}]}
      {\Gamma \vdash e : \tau & \text{subsumes}(\tau, \text{iri})}
\]

The $\text{subsumes}$ check queries the ontology at compile time but generates
no runtime code.

\section{Epistemic Runtime Representations}

For epistemic types $\Knowledge[\tau, \varepsilon, \delta, \Phi]$, we define
three runtime representation modes based on compile-time usage analysis.

\subsection{Representation Modes}

\begin{definition}[Epistemic Mode]
\begin{align*}
\Full &: 64 \text{ bytes overhead} \\
\Compact &: 16 \text{ bytes overhead} \\
\Erased &: 0 \text{ bytes overhead}
\end{align*}
\end{definition}

\subsection{Mode Selection}

The compiler selects the mode based on usage:

\begin{enumerate}
\item $\Erased$: When confidence is statically known to be $\geq 1 - 10^{-9}$
      and provenance is not accessed.
\item $\Compact$: When only point confidence and provenance hash are needed.
\item $\Full$: When confidence intervals, full provenance chains, or temporal
      metadata are accessed.
\end{enumerate}

\subsection{Full Representation}

\begin{definition}[Full Knowledge Layout]
\begin{verbatim}
struct FullKnowledge<T> {       // 64 bytes aligned
    value: T,                   // sizeof(T)
    confidence: (f64, f64, f64),// 24 bytes (point, lower, upper)
    provenance: *Provenance,    // 8 bytes
    timestamp: u64,             // 8 bytes
    flags: u64,                 // 8 bytes
    _reserved: [u8; N],         // padding to 64
}
\end{verbatim}
\end{definition}

\subsection{Compact Representation}

\begin{definition}[Compact Knowledge Layout]
\begin{verbatim}
struct CompactKnowledge<T> {    // 16 bytes overhead
    value: T,                   // sizeof(T)
    confidence_q: u16,          // 2 bytes (quantized)
    provenance_hash: u32,       // 4 bytes
    timestamp_delta: u16,       // 2 bytes
    _reserved: [u8; 8],         // 8 bytes
}
\end{verbatim}
\end{definition}

Confidence is quantized to $[0, 65535]$ mapping to $[0.0, 1.0]$ with
precision $\approx 0.0015\%$.

\subsection{Mode Subtyping}

The representations form a subtyping lattice:
\[
\Full <: \Compact <: \Erased
\]

Conversion rules:
\begin{align*}
\text{to\_compact} &: \Full \to \Compact \quad \text{(lossy)} \\
\text{to\_full} &: \Compact \to \Full \quad \text{(recovers defaults)} \\
\text{erase} &: \Compact \to \Erased \quad \text{(drops metadata)}
\end{align*}

\section{Memory Model}

\subsection{Tiered Storage}

\begin{definition}[Storage Tiers]
\begin{align*}
L_1 &: \text{HashMap}, \leq 10\text{K entries}, \leq 2\text{MB} \\
L_2 &: \text{Compact array}, \leq 100\text{K entries}, \leq 20\text{MB} \\
L_3 &: \text{Memory-mapped}, \leq 15\text{M entries}, \leq 2\text{GB}
\end{align*}
\end{definition}

\subsection{Access Complexity}

\begin{theorem}[Lookup Complexity]
\begin{align*}
\text{lookup}_{L_1} &= O(1), \text{ expected } < 100\text{ns} \\
\text{lookup}_{L_2} &= O(1), \text{ expected } < 1\mu\text{s} \\
\text{lookup}_{L_3} &= O(1), \text{ expected } < 10\mu\text{s}
\end{align*}
\end{theorem}

\subsection{Promotion Policy}

Terms are promoted between tiers based on access frequency:
\[
\text{promote}(t, L_i) =
\begin{cases}
L_{i-1} & \text{if access\_count}(t) \geq \theta \\
L_i & \text{otherwise}
\end{cases}
\]

where $\theta = 3$ is the promotion threshold.

\section{Compact Term Representation}

\begin{definition}[CompactTerm]
Each ontological term fits in exactly 64 bytes (one cache line):
\begin{verbatim}
#[repr(C, align(64))]
struct CompactTerm {
    iri_hash: u64,          // 8 bytes
    iri_data: IriStorage,   // 24 bytes (inline or Arc)
    label_data: LabelStorage,// 16 bytes (inline or Arc)
    parent_idx: u32,        // 4 bytes
    depth: u16,             // 2 bytes
    flags: u16,             // 2 bytes
    child_count: u32,       // 4 bytes
    embedding_offset: u32,  // 4 bytes
}
\end{verbatim}
\end{definition}

\section{GPU Memory Layouts}

For vectorized operations on epistemic data, we support two layouts:

\subsection{Structure of Arrays (SoA)}

\begin{definition}[SoA Layout]
\begin{verbatim}
struct SoAKnowledge<T> {
    values: Vec<T>,
    confidences: Vec<u16>,
    provenance_hashes: Vec<u32>,
    timestamps: Vec<u16>,
}
\end{verbatim}
\end{definition}

Enables coalesced memory access on GPUs.

\subsection{Array of Structures (AoS)}

\begin{definition}[AoS Layout]
\begin{verbatim}
struct AoSKnowledge<T> {
    elements: Vec<AoSElement<T>>,
}
\end{verbatim}
\end{definition}

Better cache locality for sequential single-element access.

\section{Correctness Guarantees}

\begin{theorem}[Erasure Soundness]
If $\Gamma \vdash e : \tau$ and $e$ contains ontological annotations
$O_1, \ldots, O_n$, then:
\begin{enumerate}
\item The annotations do not affect runtime behavior
\item Memory usage is bounded by $|\tau|$, not $|\tau| + \sum |O_i|$
\item Type safety is preserved after erasure
\end{enumerate}
\end{theorem}

\begin{theorem}[Multiplicity Preservation]
If $\Gamma \vdash_\mult e : \tau$ and $e \to^* v$, then the resources in
$\Gamma$ are used exactly according to their multiplicities during evaluation.
\end{theorem}

\section{Performance Targets}

\begin{center}
\begin{tabular}{|l|r|l|}
\hline
\textbf{Metric} & \textbf{Target} & \textbf{Rationale} \\
\hline
L1 lookup & $< 100$ns & Cache-resident HashMap \\
L3 lookup & $< 10\mu$s & Memory-mapped with index \\
Term creation & $< 50$ns & Inline string optimization \\
Multiplicity check & $< 5$ns & Single comparison \\
Erasure analysis & $< 1$ms/1K & Batch processing \\
Memory/term & $\leq 64$ bytes & Cache line aligned \\
\hline
\end{tabular}
\end{center}

\section{Conclusion}

This specification establishes the formal foundations for Sounio's L0
type system, enabling 15M+ ontological types with zero runtime overhead
through Quantitative Type Theory and precise erasure semantics.

\end{document}
